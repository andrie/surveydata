\documentclass[a4paper]{article}

%\VignetteIndexEntry{Introduction to surveydata package}
%\VignettePackage{surveydata}

% Definitions
%\newcommand{\slan}{{\tt S}}
%\newcommand{\rlan}{{\tt R}}
\newcommand{\sdata}{{\tt surveydata}}
\newcommand{\code}[1]{{\tt #1}}
\setlength{\parindent}{0in}
\setlength{\parskip}{.1in}
\setlength{\textwidth}{140mm}
\setlength{\oddsidemargin}{10mm}

\title{Introduction to the \sdata{} package for working with survey data.}
\author{Andrie de Vries}

\usepackage{Sweave}
\begin{document}
\Sconcordance{concordance:surveydata.tex:surveydata.Rnw:%
1 18 1 1 0 41 1 1 2 4 0 1 2 2 1 1 12 14 0 1 3 5 1 1 10 11 0 1 2 3 1 1 2 %
4 0 1 2 11 1 1 3 10 0 1 1 10 0 1 2 2 1 1 3 9 0 1 1 10 0 1 2 4 1 1 2 7 0 %
1 1 5 0 1 1 6 0 1 2 4 1 1 2 6 0 1 1 5 0 1 2 5 0 1 1 6 0 1 2 16 1}


\maketitle

\sdata{} is a package that makes it easy to work with typical survey data that originated in SPSS or other formats.

A surveydata object consists of:

\begin{itemize}

  \item{A data frame with a row for each respondent and a column for each question.   Column names are typically names in the pattern Q1, Q2\_1, Q2\_2, Q3 - where underscores separate the subquestions when these originated in a grid (array) of questions.}

  \item{Question metadata gets stored in the \texttt{variable.labels} attribute of the data frame. This typically contains the original questionnaire text for each question.}
  
  \item{Information about the subquestion separator (typically an underscore) is stored in the \texttt{patterns} attribute.}
  
\end{itemize}


Data processing a survey file can be tricky, since the standard methods for dealing with data frames does not conserve the \texttt{variable.labels} attribute.  The \sdata package defines a \texttt{surveydata} class and the following methods that knows how to deal with the \texttt{variable.labels} attribute:

\itemize{
  \item {\texttt{as.surveydata}}
  \item {\texttt{[.surveydata}}
  \item {\texttt{[<-.surveydata}}
  \item {\texttt{\$.surveydata}}
  \item {\texttt{\$<-.surveydata}}
  \item {\texttt{merge.surveydata}}
}

In addition, \sdata defines the following convenient methods for extracting and working with the variable labels:

\itemize{
  \item {varlabels}
  \item{varlabels<-}
}

\section{Example}

Load the package.

\begin{Schunk}
\begin{Sinput}
> library(surveydata)
\end{Sinput}
\end{Schunk}

Create sample data and a surveydata object:

\begin{Schunk}
\begin{Sinput}
> sdat <- data.frame(
+     id   = 1:4,
+     Q1   = c("Yes", "No", "Yes", "Yes"),
+     Q4_1 = c(1, 2, 1, 2), 
+     Q4_2 = c(3, 4, 4, 3), 
+     Q4_3 = c(5, 5, 6, 6), 
+     Q10 = factor(c("Male", "Female", "Female", "Male")),
+     crossbreak  = c("A", "A", "B", "B"), 
+     weight      = c(0.9, 1.1, 0.8, 1.2)
+ )
> varlabels(sdat) <- c(
+     "RespID",
+     "Question 1", 
+     "Question 4: red", "Question 4: green", "Question 4: blue", 
+     "Question 10",
+     "crossbreak",
+     "weight"
+   )
\end{Sinput}
\end{Schunk}


Now create a surveydata object.

\begin{Schunk}
\begin{Sinput}
> sv <- as.surveydata(sdat, renameVarlabels=TRUE)
\end{Sinput}
\end{Schunk}

It is easy to extract specific questions:

\begin{Schunk}
\begin{Sinput}
> sv[, "Q1"]
\end{Sinput}
\begin{Soutput}
   Q1
1 Yes
2  No
3 Yes
4 Yes
\end{Soutput}
\begin{Sinput}
> sv[, "Q4"]
\end{Sinput}
\begin{Soutput}
  Q4_1 Q4_2 Q4_3
1    1    3    5
2    2    4    5
3    1    4    6
4    2    3    6
\end{Soutput}
\end{Schunk}

The extraction makes use of the underlying metadata, contained in the \texttt{varlabels} and \texttt{pattern} attributes:

\begin{Schunk}
\begin{Sinput}
> varlabels(sv)
\end{Sinput}
\begin{Soutput}
                 id                  Q1                Q4_1                Q4_2 
           "RespID"        "Question 1"   "Question 4: red" "Question 4: green" 
               Q4_3                 Q10          crossbreak              weight 
 "Question 4: blue"       "Question 10"        "crossbreak"            "weight" 
\end{Soutput}
\begin{Sinput}
> pattern(sv)
\end{Sinput}
\begin{Soutput}
$sep
[1] "_"

$exclude
[1] "other"
\end{Soutput}
\end{Schunk}

It is easy to query the surveydata object to find out which questions it contains, as well as which columns store the data for those questions.

\begin{Schunk}
\begin{Sinput}
> questions(sv)
\end{Sinput}
\begin{Soutput}
[1] "id"         "Q1"         "Q4"         "Q10"        "crossbreak"
[6] "weight"    
\end{Soutput}
\begin{Sinput}
> which.q(sv, "Q1")
\end{Sinput}
\begin{Soutput}
[1] 2
\end{Soutput}
\begin{Sinput}
> which.q(sv, "Q4")
\end{Sinput}
\begin{Soutput}
[1] 3 4 5
\end{Soutput}
\end{Schunk}

The function \texttt{qText} gives access to the questionnaire text.

\begin{Schunk}
\begin{Sinput}
> qText(sv, "Q1")
\end{Sinput}
\begin{Soutput}
[1] "Question 1"
\end{Soutput}
\begin{Sinput}
> qText(sv, "Q4")
\end{Sinput}
\begin{Soutput}
[1] "Question 4: red"   "Question 4: green" "Question 4: blue" 
\end{Soutput}
\begin{Sinput}
> qTextCommon(sv, "Q4")
\end{Sinput}
\begin{Soutput}
[1] "Question 4"
\end{Soutput}
\begin{Sinput}
> qTextUnique(sv, "Q4")
\end{Sinput}
\begin{Soutput}
[1] "red"   "green" "blue" 
\end{Soutput}
\end{Schunk}


\section{Final thoughts}

The last word.


% Start a new page
% Not echoed, not evaluated
% ONLY here for checkVignettes so that all output doesn't
% end up on one enormous page

\end{document}




